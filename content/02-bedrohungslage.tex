%! Author = Gian Flütsch
%! Date = 21. Jun 2022
%! Project = IncResp Zusammenfassung

\section{Aktuelle Bedrohungslage}

\subsection{Vishing}
Beim Vishing (Voice Phishing) werden Personen mündlich zu Handlungen aufgefordert, von denen sie glauben, sie seien in ihrem Interesse. \href{https://www.kaspersky.de/resource-center/definitions/vishing}{Vishing} setzt oft da an, wo Phishing an seine Grenzen stößt.

\subsection{Vishing + E-Mail Phishing}

Oft fängt das ganze mit Phishing (z.B. via E-Mail) an und es weitet sich schlussendlich ins Vishing aus.\\

\textbf{Beispiel:}\\
Jemand besucht eine Social-Media-Plattform, klickt auf einen verlockenden Link – und schon erscheint ein blauer Bildschirm mit einer Warnmeldung und der Aufforderung, bei der angezeigten gebührenfreien Telefonnummer anzurufen, um ein ernsthaftes Problem mit dem Computer zu beheben.

Am Telefon meldet sich ein freundlicher Techniker, der gerne bereit ist zu helfen – allerdings nur gegen Bezahlung. Nachdem für den Kauf der Software, mit der das Computerproblem behoben werden soll, die Kreditkartendaten zur Verfügung gestellt wurden, ist der Betrug komplett und kommt das Opfer teuer zu stehen.

Die Software funktioniert nicht, und vom hilfsbereiten Techniker wird man nie wieder etwas hören. Der Benutzer ist ein weiteres Opfer der als „Vishing“ bezeichneten Betrugsmethode geworden.

\subsection{Mögliche Folgen}
\begin{itemize}
    \item eBanking Trojaner wird installiert
    \item Zukünftige eBanking-Aktivitäten können durch die Cyberkriminellen manipuliert werden
    \begin{itemize}
        \item Betrag ändern, Zielkonto ändern, SMS-Verifikation wird ausgehebelt
    \end{itemize}
    \item Aktuelle Antivirussoftware konnte die Schadsoftware nicht identifizieren
    \item Hätte jede andere Schadsoftwareart sein können!
\end{itemize}

\subsection{Ransomware}
\href{https://www.cert.govt.nz/business/guides/protecting-from-ransomware/}{How Ransomware works}!

\subsubsection{Ransomware Angriffe}
\begin{enumerate}
    \item Kompromittierung
    \item Sensible Daten entwenden
    \item \textcolor{red}{Daten verschlüsseln (1. Erpressung)}
    \item \textcolor{red}{Mit Veröffentlichung der Daten drohen (2. Erpressung)}
    \item Optional: Öffentlich an den Pranger stellen
    \item \textcolor{red}{Optional: Mit DDoS drohen (3. Erpressung)}
    \item \textcolor{red}{Assoziierten Personen drohen (4. Erpressung)}
    \item Optional: Business E-Mail Compromise (BEC), Phishing etc.
    \item Optional: Veröffentlichung der gestohlenen Daten
\end{enumerate}

\subsection{Wie schützen wir uns}
\begin{itemize}
    \item Mehrere, nacheinander gelagerte Schutzmechanismen (Defense in depth)
    \item Vertrauen nicht einem einzigen Produkt und Mechanismus (z.B. AV-SW, FW)
    \item 100\% Sicherheit gibt es nicht, aber:
    \begin{itemize}
        \item Wir können die Kosten für Angreifer erhöhen
        \item Wir können Angreifer verlangsamen
        \item Wir können Angreifer erkennen
        \item darum gibt es Incident Response
    \end{itemize}
\end{itemize}