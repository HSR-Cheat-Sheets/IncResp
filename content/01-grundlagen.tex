%! Author = Gian Flütsch
%! Date = 21. Jun 2022
%! Project = IncResp Zusammenfassung

\section{Incident Response Grundlagen}

\subsection{Definition}
Incident Response are actions taken to mitigate or resolve an information security incident, including those taken to protect and restore the normal operational conditions of an information system and the information stored in it.

\begin{itemize}
    \item Incident Response ist die Aktivität einen Informationssicherheitsvorfall zu behandeln
    \item Ein Vorfall ist ein oder mehrere Informationssicherheitsereignisse, die (wahrscheinlich) zu einem Schaden für die Organisation führen
    \item Ein Ereignis verletzt die Aktivitäten eines Unternehmens zur Sicherstellung der Informationssicherheit
    \begin{itemize}
        \item nicht nur FW deaktivieren etc. $\rightarrow$ MA, welcher NB entsperrt liegen lässt kann auch ein Security Incident werden
    \end{itemize}
    \item Incident Response ist die Bewältigung einer Verletzung der Informationssicherheit
\end{itemize}

\subsection{Information Security Incident (Informationssicherheitsvorfall)}
Einzelnes oder eine Reihe von ungewollten oder unerwarteten Informationssicherheitsereignissen, die eine erhebliche Wahrscheinlichkeit besitzen, Geschäftstätigkeiten zu gefährden und die Informationssicherheit zu bedrohen.

\subsection{Information Security Event (Informationssicherheitserreignis)}
Erkanntes Auftreten eines Zustands eines Systems, Dienstes oder Netzwerks, der eine mögliche Verletzung der Politik oder die Unwirksamkeit von Massnahmen oder eine vorher nicht bekannte Situation, die sicherheitsrelevant sein kann, anzeigt.

Bei einem \textbf{\textcolor{OSTPink}{Ereignis}} kann etwas vorhanden sein (z.B. AV Meldung $\rightarrow$ true positive oder falscher Alarm?) $\rightarrow$ falls Mimikatz in AV Report steht $\rightarrow$ befindet man sich schon im \textbf{\textcolor{OSTPink}{Event}} und nicht mehr im \textbf{Ereignis}

Bei einem \textbf{\textcolor{OSTPink}{Event}} ist wirklich etwas (effektiver Security Vorfall)!.

\subsection{Ziel Informationssicherheit}
CIA Triad $\rightarrow$ Confidentiality, Integrity, Availability

\subsubsection{Confidentiality}
Information wird unbefugten nicht verfügbar gemacht oder offengelegt.

\subsubsection{Integrity}
Information ist richtig und vollständig.

\subsubsection{Availability}
Information ist für eine befugte Entität bei Bedarf zugänglich.

\subsection{Schützenswerte Daten}
\begin{itemize}
    \item Kundendaten $\rightarrow$ DSG (Datenschutz Gesetz)/ GDPR
    \item Mitarbeiterdaten $\rightarrow$ DSG
    \item PII/ PHI $\rightarrow$ DSG
    \item Backups
    \item Trade Secrets
\end{itemize}

% TODO: siehe Übung W01 + Quiz 1