\section{Verhandlung mit Ransomware-Gruppen}

\subsection{Ablauf Ransomware}
\begin{enumerate}
    \item Kompromittierung
    \item Diebstahl sensibler Daten
    \item \textcolor{OSTPink}{Daten verschlüsseln (1. Erpressung)}
    \item \textcolor{OSTPink}{Mit Veröffentlichung der Daten drohen (2. Erpressung)}
    \item Optional: Öffentlich an den Pranger stellen
    \item \textcolor{OSTPink}{Optional: Androhung von DDoS-Angriffen (3. Erpressung)}
    \item \textcolor{OSTPink}{Optional: Kunden/ Nutzer/ Mitarbeiter drohen (4. Erpressung)}
    \item Optional: Business Email Compromise (BEC), Phishing, etc.
    \item Optional: Veröffentlichung gestohlener Daten
\end{enumerate}

\subsection{Grundlage für die Verhandlung}
\begin{itemize}
    \item Ransomware-Betreiber haben ein wirtschaftliches Interesse daran, sich an ihre Vereinbarung zu halten
    \item Unternehmen zahlen, weil sie wissen, dass sie mit hoher Wahrscheinlichkeit das erhalten, was versprochen wird
    \begin{itemize}
        \item Typisches Argument betroffener Unternehmen für eine Zahlung
    \end{itemize}
    \item Sie haben ein hohes Interesse daran, eine Zahlung zu leisten. Die Veröffentlichung hat einen geringen wirtschaftlichen Wert
    \item Sie üben Druck aus mit
    \begin{itemize}
        \item der Drohung mit der Veröffentlichung
        \item kurzen Zeitfenstern für Entscheidungen
        \item Variationen im Kommunikationsstil
    \end{itemize}
\end{itemize}

\subsection{Sollen wir verhandeln?}
\begin{itemize}
    \item Meistens: Ja! Unabhängig vom Wunsch Geld zu bezahlen
    \begin{itemize}
        \item \textbf{Vorsicht}: Veröffentlichung der Konversation könnte von der Öffentlichkeit missverstanden werden
    \end{itemize}
    \item Ziele:
    \begin{itemize}
        \item Alle Möglichkeiten offen halten
        \item Zeit für den Aufbau der Verteidigung / der Schutzmassnahmen gewinnen
    \end{itemize}
\end{itemize}

\subsection{Verhandlung}
\begin{itemize}
    \item Festlegen, was man von der Gegenseite als Gegenleistung haben möchte
    \begin{itemize}
        \item Was sind unsere Ziele?
    \end{itemize}
    \item Maximalbetrag und Zielbetrag festlegen
    \begin{itemize}
        \item Wenn man eine Cyberversicherung hat: Könnte ggf. von der Cyberversicherung gedeckt sein
    \end{itemize}
    \item Sich stets die Zeit nehmen eine Nachricht zu verarbeiten und darauf zu antworten
    \item Wenn möglich mit Bitcoins bezahlen
    \begin{itemize}
        \item Ist für die Strafverfolgung einfacher zu verfolgen
    \end{itemize}
    \item Juristen sollten involviert werden
    \begin{itemize}
        \item Gewisse Ransomware-Gruppen sind sanktioniert und es kann zu zusätzlichen negativen Folgen kommen
    \end{itemize}
\end{itemize}

\subsection{Grundlagen für Verhandlung}
\begin{itemize}
    \item Der Preis ist verhandelbar! Grundsätzlich und durch Argumente, wie zum Beispiel:
    \begin{itemize}
        \item Unternehmensgrösse wurde falsch eingeschätzt
        \item Umsatz/Gewinn wurde falsch bewertet
        \item Aktuelle ökonomische Situation bzw. Prognosen für die Zukunft
    \end{itemize}
    \item Wir spielen grundsätzlich immer auch auf Zeit
    \begin{itemize}
        \item Verteidigung ausbauen
        \item Was wenn der Angreifer noch Zugang hat und wir haben diesen noch nicht entdeckt und entfernt?
        \item Teilzahlungen machen und bei jeder eine Gegenleistung verlangen
    \end{itemize}
    \item Ideen, was man als Gegenleistung verlangen kann:
    \begin{itemize}
        \item Entschlüsselung einer oder mehrerer Dateien als Fähigkeitsnachweis
        \item Entschlüsselungsprogramm
        \item Bericht, in dem beschrieben wird, wie sie eingedrungen sind, wie und wo sie sich eingenistet haben, was sie ausgenutzt haben usw.
        \item Anweisungen zum Entfernen der Schadsoftware
        \item Nachweis der gestohlenen Daten (z. B. Verzeichnisliste)
        \item Beweis für die Löschung von Daten
    \end{itemize}
\end{itemize}

\subsection{Beispiele}
$\rightarrow$ siehe Vorlesung Folien 17 - 47.