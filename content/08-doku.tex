\section{Dokumentation}

\subsection{Gründe für Dokumentation}
\begin{itemize}
    \item Fortlaufend den Überblick behalten
    \item Entscheidungsfindung ermöglichen und unterstützen
    \item Lessons Learned ermöglichen
    \begin{itemize}
        \item Was ist passiert?
        \item Wer war wie und wann involviert?
        \item Was wurde herausgefunden/festgestellt?
        \item Welche Massnahmen wurden aus welchem Grund und wann umgesetzt?
        \item Was kann man in Zukunft verbessern?
    \end{itemize}
\end{itemize}

\subsection{Während dem Vorfall}
\begin{itemize}
    \item Generell immer:
    \begin{itemize}
        \item Zeitstempel des Zeitpunkts als es festgestellt/gefunden/dokumentiert wurde
        \item Zeitstempel des Ereignisses selbst (wann hat es stattgefunden?)
        \item Person, Ort des Fundes (System, Pfad, IP-Adressen etc.)
    \end{itemize}
    \item Welche Systeme sind involviert?
    \begin{itemize}
        \item Hostname, IP-Adresse, relevante Ereignisse
    \end{itemize}
    \item Was wurde wo gemacht?
    \item Aufgabenübersicht
    \begin{itemize}
        \item Was wurde wann durch wen gemacht?
        \item Was ist in welchem Durchführungszustand?
        \item Priorisierung
        \item Warum wurde die Aufgabe aufgenommen? Was ist der Hintergrund/Kontext?
    \end{itemize}
\end{itemize}

\subsection{Nach dem Vorfall}
\paragraph{Einführung}
\begin{itemize}
    \item Beschreibung des Vorfalls auf hoher Ebene
    \item Einstufung / Kategorisierung des Vorfalls
    \item Zeitlicher Ablauf (auf organisatorischer Ebene)
    \item Involvierte Personen und deren Rollen / Aufgaben (Projektorganisation)
    \item Hypothesen oder zu beantwortende Fragen
\end{itemize}

\paragraph{Lösungsansatz/ Arbeitsweise/ Vorgehen}
\begin{itemize}
    \item Allgemeine Beschreibung des Incident-Response-Prozesses
    \item Eingesetzte Werkzeuge
\end{itemize}

\paragraph{Incident Response}
\begin{itemize}
    \item Übersicht der Massnahmen und deren Status (z. B. geplant, erledigt, abgelehnt)
    \item Verlauf des Informationssicherheitsvorfalls
    \begin{itemize}
        \item Wann haben welche Personen welche Entscheidungen getroffen auf Basis welcher Informationen?
        \item Was wurde als die nächsten Schritte festgelegt
    \end{itemize}
    \item Details
\end{itemize}

\paragraph{IT-Forenische Analyse}
\begin{itemize}
    \item Übersicht: Kernereignisse als Ablauf / Zeitstrahl mit Zeitstempel (UTC), Ereignisbeschreibung und Verweis zu den Details
    \item Übersicht untersuchter Systeme
    \item Pro Werkzeug, Untersuchungsschritt, Asset und/oder Artefakt die Untersuchung im Detail beschreiben
    \item Pro Angriffsschritt die Feststellungen dokumentieren
\end{itemize}

\paragraph{Fazit}
\begin{itemize}
    \item Hypothesen belegen/wiederlegen basierend auf den gesammelten und analysierten Artefakten
    \item Fragen beantworten
    \item Empfehlungen für die Zukunft (Lessons Learned)
\end{itemize}

\paragraph{Anhang}
\begin{itemize}
    \item Hashes (Aller Images \& untersuchten Dateien)
    \item IOCs (Kurzbeschreibung des IOC + Verweis auf relevantes Detailkapitel)
    \item Screenshots \& Fotos
    \item Kopie relevanter Formulare
\end{itemize}

\subsection{Häufige Fehler und Misskommunikation}
\begin{itemize}
    \item Zeitstempel ohne Kontext
    \begin{itemize}
        \item Auf was bezieht sich der Zeitstempel? Was ist zu diesem Zeitpunkt passiert?
    \end{itemize}
    \item Zeitstempel ohne Zeitzone
    \item Vermischung von Analyseresultaten und deren Interpretation
    \begin{itemize}
        \item Resultat einer (forensischen) Untersuchung eines Artefakt sind für sich alleine objektiv zu beschreiben
        \item Aussagekraft kann durch Kombination zusammenhängender Artefakte gesteigert werden
        \item Was wurde auf einem System ausgeführt, welche Hinweise wurden entdeckt, etc.
        \item Interpretation der Resultate im Kontext des Vorfalls / Angriffs klar vom Rest trennen und mit Wahrscheinlichkeiten arbeiten
        \item Im Normalfall gibt es immer mehrere Erklärungen für ein Resultat!
    \end{itemize}
\end{itemize}
