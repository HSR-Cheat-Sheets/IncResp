\section{Digital Forensics}

\subsection{Definition}
\begin{itemize}
    \item Ein streng methodischer Ansatz zur Analyse von Daten
    \item Die Untersuchung folgt einem gerichtsfestem Ansatz
    \begin{itemize}
        \item Die Integrität der Daten wird sichergestellt
        \item Wissenschaftlich unvoreingenommene Analyse
        \item Grundsätzlich sind geprüfte und akzeptierte Verfahren und Werkzeuge für die Sammlung, Aufbewahrung und Analyse im Einsatz
        \item Die Ergebnisse müssen bei Bedarf durch Dritte nachvollzogen und reproduziert werden
    \end{itemize}
    \item \textbf{\textcolor{red}{Digitale Forensik ist die Sicherung, Aufbereitung und Analyse digitaler Spuren in einer Weise, die in einer späteren Gerichtsverhandlung akzeptiert wird}}
\end{itemize}

\subsection{Methoden}

\begin{minipage}{0.4\linewidth}
    \paragraph{Live-Forensik}
    \begin{itemize}
        \item Wird auf dem laufenden IT-System durchgeführt
        \item Flüchtige Daten können gesammelt werden
        \item Beweissicherung kann zu unerwünschten Änderungen am System führen
        \item Tool: \lstinline|KAPE|
    \end{itemize}
    $ $
\end{minipage}
\begin{minipage}{0.5\linewidth}
    \paragraph{Dead-Forensik}
    \begin{itemize}
        \item Wird auf dem ausgeschalteten IT-System durchgeführt
        \item Sammlung kann ohne Änderungen an den Daten vorgenommen werden
    \end{itemize}
\end{minipage}

\subsubsection{Live-Forensik (Online-Forensik oder Live Response)}
\begin{itemize}
    \item Untersuchung während der Laufzeit
    \item Erlaubt es flüchtige Daten zu sammeln und damit diese zu untersuchen
    \begin{itemize}
        \item Inhalt des Arbeitsspeichers, Swap / Page Dateien
        \item Informationen zu bestehenden Netzwerkverbindungen
        \item Informationen zu gestarteten Prozessen
        \item Informationen zu offene Dateien, Sockets, Pipes etc.
        \item Angemeldete Nutzerkonten
    \end{itemize}
    \item Damit kann der aktuelle Laufzeitzustand des Systems festgehalten und untersucht werden (z.B. Autoruns mit KAPE)
    \item Die Reihenfolge, in der Daten gesammelt werden, ist sehr wichtig!
    \begin{itemize}
        \item Je flüchtiger die Daten sind, desto schneller müssen sie angesehen werden!!!
        \item Je schneller sich Daten ändern bzw. durch Systemaktivitäten verändert werden, desto flüchtiger sind diese
    \end{itemize}
    \item Daten werden in der Reihenfolge ihrer Flüchtigkeit gesammelt: Flüchtigste Daten zuerst
    \begin{enumerate}
        \item \textcolor{red}{Arbeitsspeicher}
        \item \glqq Auslagerungsdateien\grqq{}: Swap / Page Datei(en)
        \item Netzwerkstatus, Netzwerkverbindungen (Caches nicht vergessen)
        \item Laufende Prozesse
        \item Offene Dateien
        \item ...
    \end{enumerate}
\end{itemize}

\subsubsection{Dead-Forensik}
Berühre, verändere oder modifiziere niemals etwas, bevor es nicht dokumentiert, gekennzeichnet, vermessen und fotografiert ist.

\begin{itemize}
    \item Untersuchung eines nicht aktiven Systems
    \item Untersuchung von Daten, die nach dem deaktivieren/ausschalten eines IT-Systems zur Verfügung stehen
    \begin{itemize}
        \item Fokus auf die nicht-flüchtigen Datenträger
    \end{itemize}
    \item Findet häufig \underline{nach} einem Vorfall statt oder wenn dieser bereits lange her ist
\end{itemize}

\subsection{Datensammlung}
\begin{itemize}
    \item Unabhängig von der Art der Datensammlung, müssen wir zuerst grundlegende Informationen festhalten (dokumentieren!)
    \begin{itemize}
        \item Was wird gesammelt?, Von welchem System wird es gesammelt?
        \item Von welchem Datenträger stammen die Daten?, Wer hat die Daten gesammelt?
        \item Wann wurde die Datensammlung gestartet? Wann wurde sie abgeschlossen?
        \item Was ist über das System bekannt? Systemzeit, Zeitzone des Systems etc.
        \item Wie und mit welchen Werkzeugen wurden die Daten gesammelt?
    \end{itemize}
    \item \textbf{\textcolor{red}{Veränderung der Originaldaten muss zwingend vermieden werden!}}
    \item \textbf{\textcolor{red}{Bewehrte Werkzeuge und Vorgehen verwenden (forensically sound)}}
    \item \textbf{\textcolor{red}{Die präzise Dokumentation ist absolut zentral und notwendig}}
\end{itemize}

\subsubsection{Full Image}
\begin{itemize}
    \item Ein Datenträger wird vollständig kopiert (Kopie ist bitweise / sektorweise)
    \item Damit werden auch vom System nicht genutzte Speicherbereiche, gelöschte Dateien etc. mitkopiert
    \item Kann \underline{nicht} mit Standardwerkzeugen üblicher Betriebssystem (Windows, macOS, Linux) durchgeführt werden
\end{itemize}

Beim erstellen eines Full Image einen \textit{Write Blocker} dazwischen schliessen, dass alle Write-Befehle des Clients an die zu kopierende Festplatte geblockt werden.
Damit kann gewährleistet werden, dass sich die Festplatte nicht während oder wegen dem kopieren ändert!

\subsubsection{Memory Image}
\begin{itemize}
    \item Sicherung des Arbeitsspeichers
    \begin{itemize}
        \item In alltagstauglichen Fällen nur bei Live-Forensik
    \end{itemize}
    \item \textbf{Wichtig}: Werkzeug für die forensische Sicherung des Arbeitsspeichers nutzen
    \begin{itemize}
        \item Sollte selbst minimalen Arbeitsspeicher einnehmen und soweit wie möglich keine Änderungen am System vornehmen
        \item  Tool: \lstinline|volatility|
    \end{itemize}
\end{itemize}

\subsubsection{Triage Image (Triage-Forensik)}
\begin{itemize}
    \item Daten werden zielgerichtet gesammelt
    \begin{itemize}
        \item Fokus auf typischerweise forensisch relevante Dateien
    \end{itemize}
    \item Beispiele
    \begin{itemize}
        \item Historie, Cache, Favoriten, Download-Historie, Cookies etc. von Browsern
        \item Logdateien, Registry, Prefetch, Shimcache, Amcache
        \item Gelöschte Dateien (Recycle Bin)
    \end{itemize}
    \item Tool: \textit{KAPE}
\end{itemize}

\subsubsection{Prefetch}
\begin{itemize}
    \item \lstinline|C:\Windows\Prefetch\|
    \item Proof of Execution
    \item Remains after deletion of the Executable
    \item Shows loaded DLLs (first 10 sec. of execution are logged)
    \item Goal is to reduce the start time of the application
    \item ist grundsätzlich auf Servern deaktiviert
\end{itemize}

\subsubsection{AMCache}
\begin{itemize}
    \item \lstinline|C:\Windows\AppCompat\Programs\|
    \item Registry-Hive
\end{itemize}

\subsubsection{SHIMCache (AppCompatCache/ Application Compatability Cache)}
\begin{itemize}
    \item \lstinline|HKEY_LOCAL_MACHINE\SYSTEM\CurrentControlSet\Control\Session Manager\AppCompatCache|
    \item Provides compatibility for older software running in newer versions of Windows (backward compatibility)
    \item Executable file name, file path \& timestamp are recorded (timestamp = last modification time)
    \item Stored in the SYSTEM registry hive
    \item Seit Win10 kann Shimcache nicht mehr genutzt werden um zu sagen, ob ein Programm ausgeführt wurde oder nicht!!
    \begin{itemize}
        \item Aber es kann gezeigt werden, dass ein File einmal auf dem System existiert hat oder was über ein externes Laufwerk/ UNC-Pfad angesteuert wurde
    \end{itemize}
    \item Only written on reboot or shutdown
    \item Shimcache can be used to show executable files present on, or accessed via a given system
\end{itemize}