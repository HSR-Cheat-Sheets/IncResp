\section{Darknet}

\subsection{Begriffe}
\begin{minipage}{0.6\linewidth}
    \paragraph{Clear Web/ Surface Web}
    \begin{itemize}
        \item Klassisches Internet
        \item kann von Suchmaschinen indexiert werden
    \end{itemize}
    \paragraph{Deep Web}
    \begin{itemize}
        \item zuerst Authorisierung notwenig
        \item kann deshalb von Suchmaschinen nicht indexiert werden
        \item macht den grössten Teil aus
    \end{itemize}
    \vfill
    $ $
\end{minipage}
\begin{minipage}{0.4\linewidth}
    \paragraph{Darkweb/ Darknet}
    \begin{itemize}
        \item Jedes Darknet nutzt einen \underline{eigenen Kommunikationsstandard}
        \item Darknet spezifische Software notwendig
        \item Kann im Normalfall \underline{nicht} auf einfache Weise indexiert werden
        \item Bekanntestes Darknet: \underline{Tor-Netzwerk}
    \end{itemize}
\end{minipage}

\subsection{Tor-Netzwerk}
\begin{itemize}
    \item Nutzt im Normalfall das Internet als Kommunikations-Infrastruktur
    \begin{itemize}
        \item Tor ist ein \textit{Overlay-Netzwerk}
    \end{itemize}
    \item Primäres Ziel: Anonymisierung von Verbindungsdaten
    \begin{itemize}
        \item Kommunikation ist sehr schwer zu überwachen
        \item Quelle und Ziel einer Verbindung schwer zu überwachen
    \end{itemize}
\end{itemize}